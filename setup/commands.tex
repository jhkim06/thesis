%
% content shorthands
%

% simple table
\newcommand{\Table}[6]{
	\begin{table}[#1]
		\begin{center}
			\begin{tabular}{#5}
				#6
			\end{tabular}
			\caption[#3]{#3#4}
			\label{#2}
		\end{center}
	\end{table}
}

% simple figure
\newcommand{\Figure}[6]{
	\begin{figure}[#1]
		\begin{center}
			\includegraphics[width=#2\textwidth]{#6}
			\caption[#4]{#4#5}
			\label{#3}
		\end{center}
	\end{figure}
}

% default equation
\newcommand{\eqn}[1]{
	\begin{align}
		#1
	\end{align}
}

% unnumbered equation
\newcommand{\eqnu}[1]{
	\begin{align*}
		#1
	\end{align*}
}

% aligned equation
\newcommand{\eqnat}[2]{
	\begin{alignat}{#1}
		#2
	\end{alignat}
}

% unnumbered, aligned equation
\newcommand{\eqnatu}[2]{
	\begin{alignat*}{#1}
		#2
	\end{alignat*}
}
